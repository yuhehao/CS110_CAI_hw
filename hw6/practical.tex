%%%%%%%%%%%%%%%%%%%%%%%%%%%%%%%%%%%%%%%%%%%%%%%%%%%%%%%%%%%%%%%%%%%%%
% This is a LaTeX template for students
% working on Homework 6, Computer Architecture I, Spring, 2022.
% You SHALL NOT distribute this template.
%%%%%%%%%%%%%%%%%%%%%%%%%%%%%%%%%%%%%%%%%%%%%%%%%%%%%%%%%%%%%%%%%%%%%
\section{Let's See Some Real World Example}

Each time when you access the course website, your activity will be
recorded into our web server logs! This is the definition of the web
server log for our Computer Architecture course website. Assume our
web server is a 32-bit machine. In this question, we will examine
code optimizations to improve log processing speed. The data
structure for the log is defined below.
\begin{minted}{c}
    struct log_entry {
        int src_ip;    /* Remote IP address */
        char URL[128]; /* Request URL. You can consider 128 characters are enough. */
        long reference_time; /* The time user referenced to our website. */
        char browser[64]; /* Client browser name */
        int status; /* HTTP response status code. (e.g. 404) */
    } log[NUM_ENTRIES];
\end{minted}
Assume the following processing function for the log. This function
determines the most frequently observed source IPs during the given
hour that succeed to connect our website.

\begin{minted}{c}
    topK_success_sourceIP (int hour);
\end{minted}

\begin{questions}

\question[2] Which field(s) in a log entry will be accessed for the
given log processing function?

{
    \begin{solution}
        Note: There may exist(s) one or more correct choice(s).\\
        % If you find the correct answer, substitute `\choice`
        % by `CorrectChoice`!
        \begin{oneparcheckboxes}
            \choice \texttt{src\_ip}
            \choice \texttt{URL}
            \choice \texttt{reference\_time}
            \choice \texttt{browser}
            \choice \texttt{status}
        \end{oneparcheckboxes}
    \end{solution}
}

\question[1] Assuming 32-byte cache blocks and no prefetching, how
many cache misses per entry does the given function incur on average? \label{q:miss}

{
    \begin{solution}
        % fill your answer in \fillin arguments.
        \fillin[][4in]
    \end{solution}
}

\question[3] How can you reorder the data structure to improve
cache utilization and access locality? Justify your modification.

{
    \begin{solution}
        \vspace{1in}
        \begin{minted}{c}
            struct log_entry {
                int status; /* HTTP response status code. (e.g. 404) */
                long reference_time; /* The time user referenced to our website. */
                int src_ip;    /* Remote IP address */
                char browser[64]; /* Client browser name */
                char URL[128]; /* Request URL. You can consider 128 characters are enough. */
            } log[NUM_ENTRIES];
        \end{minted}
    \end{solution}
}

\newpage

\question[6] To mitigate the miss in the question \ref{q:miss},
design a different data structure. How would you rewrite the
program to improve the overall performance?

Your answer shall include:

\begin{itemize}
    \item A new layout of data structure of our server logs.
    \item A description of how your function would improve the
    overall performance.
    \item How many cache misses per entries does your improved
    design incur on average?
\end{itemize}

{
    \begin{solution}
        %%%%%%%%%%%%%% YOUR ANSWER HERE %%%%%%%%%%%%%%%%%%%%%%%%%
        \begin{minted}{c}
            struct log_entry {
                int status; /* HTTP response status code. (e.g. 404) */
                int src_ip;    /* Remote IP address */
                long reference_time; /* The time user referenced to our website. */
            } log[NUM_ENTRIES];
        \end{minted}
        %%%%%%%%%%%%%%%%%%%%%%%%%%%%%%%%%%%%%%%%%%%%%%%%%%%%%%%%%
        \vspace{4in}
    \end{solution}
}

\end{questions}